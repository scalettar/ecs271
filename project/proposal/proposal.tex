\documentclass[11pt]{article}
\usepackage[margin=1in]{geometry}
\usepackage{amsmath, amssymb}
\usepackage{hyperref}
\usepackage{enumitem}
\usepackage{graphicx}

% Dafny code highlighting
\usepackage{xcolor}
\usepackage{listings}

\definecolor{codegreen}{rgb}{0,0.6,0}
\definecolor{codegray}{rgb}{0.5,0.5,0.5}
\definecolor{codepurple}{rgb}{0.58,0,0.82}
\definecolor{backcolour}{rgb}{0.97,0.97,0.97}

%----------------------------------------------------------------------------------------

\title{Unsupervised Anomaly Detection in Network Traffic using Deep Autoencoders}
\author{Daniel Scalettar (scalettar@ucdavis.edu)}
\date{ECS 271 - Project Proposal}

\begin{document}

\maketitle

%----------------------------------------------------------------------------------------
%	ARTICLE CONTENTS
%----------------------------------------------------------------------------------------

\textit{Note on Group: I attempted to form a group in person and through piazza;
however, as I was unable to find any members, I have framed this as a solo project.}

\section{Problem}

This project aims to address the problem of detecting novel, "zero-day,"
network intrusions in high-volume traffic.
The core challenge is to build a model that can create a robust,
compressed representation of normal network behavior,
allowing it to detect anomalies that deviate from this norm as potential intrusions.

\section{Motivation}

Typical supervised intrusion detection systems (IDS) are limited by their reliance
on known attack signatures. The reliance on labeled data limits their ability
to detect novel, "zero-day" attacks, which are not represented in the training data.
Unsupervised anomaly-based detection systems address this limitation by learning
a robust profile of normal network behavior from unlabeled data,
and identify deviations from this profile as potential intrusions.
However, simple unsupervised linear methods like PCA may fail to capture complex
patterns in high-dimensional network traffic data.
This project is motivated by the need for more effective unsupervised models,
that can learn intricate, non-linear representations of normal network behavior.
Deep autoencoders, with their ability to learn hierarchical feature representations,
have shown promise in this regard and is an ongoing area of research. 

\section{Dataset}

This project will utilize the \textbf{CSE-CIC-IDS2018} dataset \cite{CICIDS2018}.
This 

\section{Methodology}


%------------------------------------------------

\begin{thebibliography}{9}

	\bibitem{CICIDS2018}
	Canadian Institute for Cybersecurity. (2018).
	\newblock \textit{CSE-CIC-IDS2018 on AWS (CSE-CIC-IDS2018)}.
	\newblock University of New Brunswick.
	\newblock Retrieved October 6, 2025, from \url{https://www.unb.ca/cic/datasets/ids-2018.html}.

\end{thebibliography}

\end{document}
